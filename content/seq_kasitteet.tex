\section{Käsitteitä}

% Pitäisikö mainita jono, siis ilman luku-?

% Tämä kappale pitää kirjoittaa uudestaa selkeämmin

Kun lukuja luetellaan peräkkäin jossain määrätyssä järjestyksessä, ne muodostavata lukujonon. Sama luku voi esiintyä useaan kertaan lukujonossa. Lukujonon jäseniä voidaan nimittää myös termeiksi.

Lukujonon $(a_n)$ ensimmäistä jäsentä merkitään $a_1$, toista jäsentä $a_2$ ja niin edelleen.

Lukujonon $3, 4, 5, 6, \ldots$ yleisen eli $n$. jäsenen laskemiselle voidaan esittäää erilaisa sääntöjä. Sääntöä $a_n=n+2$ sanotaan analyyttiseksi tai eksplisiittiseksi säännöksi. Tällaisen säännön avulla mikä tahansa jonon jäsen voidaan selvittää suoraan sijoituksella.

Toinen tapa jonon määrittämiseksi olisi antaa rekursiokaava $a_n=a_{n-1}+1$ ja ensimmäinen jäsen $a_1=3$.
