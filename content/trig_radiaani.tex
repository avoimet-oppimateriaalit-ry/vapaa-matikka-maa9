\section{Radiaani}
Kulman suuruuden mittarina käytetään usein asteita, joiden määritelmä on jokseenkin mielivaltainen: täyden kierroksen suuruudeksi on päätetty $360^\circ$. Nyt esitellään toinen kulman suuruuden yksikkö, radiaani.

Piirretään kulman, jonka suuruus on $\alpha$, kärkipisteen ympärille $r$-säteinen ympyrä, josta kulman kyljet erottavat kaaren, jonka pituus on $x$. Nyt jos kulmaa $\alpha$ kasvatetaan, kaarenpituus $x$ kasvaa samassa suhteessa. Kaarenpituus $x$ olisi siis muuten hyvä mitta kulman suuruudelle, mutta se riippuu valitusta säteestä $r$.

\begin{kuva}
skaalaa(5)
a = 0.24 * pi
a_deg = 180 * a / pi
P = geom.piste(0, 0)
A = (1, 0)
B = (cos(a), sin(a))
geom.jana(P, A, "$r$", 0.5 * 0.75, puoli = False)
geom.jana(P, B)
geom.kaari(P, 0.75, 0, a_deg, "$x$", 0.5 * a_deg)
geom.kulma(A, P, B, r"$\alpha$")
\end{kuva}
\begin{kuva}
skaalaa(5)
a = 0.24 * pi
a_deg = 180 * a / pi
P = geom.piste(0, 0)
A = (1, 0)
B = (cos(a), sin(a))
geom.jana(P, A, "$r'$", 0.5 * 0.9, puoli = False)
geom.jana(P, B)
geom.kaari(P, 0.9, 0, a_deg, "$x'$", 0.5 * a_deg)
geom.kulma(A, P, B, r"$\alpha$")
\end{kuva}

Jos valitaan kaarenpituudeksi jokin toinen luku $r'$, saadaan toinen kaarenpituus $x'$, mutta edelleen kulman kylkien erottamat sektori on yhdenmuotoinen alkuperäisen kanssa, joten kaaren pituuden suhde säteeseen on molemmissa sektoreissa sama:
\[\frac{x}{r} = \frac{x'}{r'}.\]

Tästä seuraa, että millä tahansa säteen $r$ valinnalla suhde $\frac{x}{r}$ säilyy vakiona samalle kulmalle $\alpha$. Siis suhde $\frac{x}{r}$ on luonnollinen kulman suuruuden mitta, jota kutsumme radiaaniksi:

\begin{luoKuva}{radiaanimaaritelma}
skaalaa(3)
a = 0.24 * pi
a_deg = 180 * a / pi
P = geom.piste(0, 0)
A = (1, 0)
B = (cos(a), sin(a))
geom.jana(P, A, "$r$", 0.5, puoli = False)
geom.jana(P, B)
geom.kaari(P, 1, 0, a_deg, "$x$", 0.5 * a_deg)
geom.kulma(A, P, B, r"$\alpha$")
\end{luoKuva}

\laatikko{
{\bf Radiaanin määritelmä}

\raisebox{-.5\height}{\naytaKuva{radiaanimaaritelma}}
Kulman $\alpha$ suuruus radiaaneina on $\displaystyle\frac{x}{r}$.
}

Kulman suuruus radiaaneina on kahden käyränpituuden suhteena paljas luku, mutta monesti merkitään kulman suuruuden perään rad korostamaan sitä, että kyseessä on kulma.

Jos määritelmässä valitaan $r = 1$, nähdään että kulman suuruus radiaaneina voidaan tulkita kulman erottaman kaaren pituudeksi yksikköympyrällä.

Täyskulma ($360^\circ$) on radiaaneina $2\pi$ rad, koska $1$-säteisen ympyrän kehän pituus on $2\pi$. Kaaren pituus on suoraan verrannollinen kulman suuruuteen, eli on olemassa reaaliluku $a$ siten, että jos kulma on suuruudeltaan $x$ radiaania ja $y$ astetta, niin
\[x = a y.\]
Täyskulman tapauksessa $x = 2\pi$ ja $y = 360$, eli $2\pi = 360 a$, josta voidaan ratkaista $a = \frac{\pi}{180}$.

\laatikko{
Kulman suuruus radiaaneina saadaan kertomalla kulman suuruuden lukuarvoa asteina luvulla $\frac{\pi}{180}$.

Kulman suuruus asteina saadaan kertomalla kulman suuruuden lukuarvoa radiaaneina luvulla $\frac{180}{\pi}$.
}

Esimerkkejä muunnoksista asteista radiaaneiksi: \\
\begin{tabular}{|l|l|}
\hline
$360^\circ$ & $2\pi$ rad \\
\hline
$180^\circ$ & $\pi$ rad \\
\hline
$120^\circ$ & $\frac{2\pi}{3}$ rad \\
\hline
$90^\circ$ & $\frac{\pi}{2}$ rad \\
\hline
$60^\circ$ & $\frac{\pi}{3}$ rad \\
\hline
$45^\circ$ & $\frac{\pi}{4}$ rad \\
\hline
$36^\circ$ & $\frac{\pi}{5}$ rad \\
\hline
$30^\circ$ & $\frac{\pi}{6}$ rad \\
\hline
\end{tabular}

\begin{tehtavasivu}
\begin{tehtava}
Muunna kulmat radiaaneiksi
\begin{alakohdat}
\alakohta{$270^\circ$}
\alakohta{$2^\circ$}
\alakohta{$10^\circ$}
\alakohta{$15^\circ$}
\alakohta{$337^\circ$}
\alakohta{$9^\circ$}
\end{alakohdat}

\begin{vastaus}
\begin{alakohdat}
\alakohta{$\frac{3\pi}{2}$}
\alakohta{$\frac{\pi}{90}$}
\alakohta{$\frac{\pi}{18}$}
\alakohta{$\frac{\pi}{12}$}
\alakohta{$\frac{337\pi}{180}$}
\alakohta{$\frac{\pi}{20}$}
\end{alakohdat}
\end{vastaus}
\end{tehtava}

\end{tehtavasivu}
