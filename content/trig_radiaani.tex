\section{Radiaani}
Kulman suuruuden mittarina käytetään usein asteita, joiden määritelmä on jokseenkin mielivaltainen: täyden kierroksen suuruudeksi on päätetty $360^\circ$. Nyt esitellään toinen kulman suuruuden yksikkö, radiaani.

Piirretään kulman, jonka suuruus on $\alpha$, kärkipisteen ympärille $r$-säteinen ympyrä, josta kulman kyljet erottavat kaaren, jonka pituus on $c$. Nyt jos kulmaa $\alpha$ kasvatetaan, kaarenpituus $c$ kasvaa samassa suhteessa. Kaarenpituus $c$ olisi siis muuten hyvä mitta kulman suuruudelle, mutta se riippuu valitusta säteestä $r$.

\begin{kuva}
skaalaa(5)
a = 0.24 * pi
a_deg = 180 * a / pi
P = geom.piste(0, 0)
A = (1, 0)
B = (cos(a), sin(a))
geom.jana(P, A, "$r$", 0.5 * 0.75, puoli = False)
geom.jana(P, B)
geom.kaari(P, 0.75, 0, a_deg, "$c$", 0.5 * a_deg)
geom.kulma(A, P, B, r"$\alpha$")
\end{kuva}
\begin{kuva}
skaalaa(5)
a = 0.24 * pi
a_deg = 180 * a / pi
P = geom.piste(0, 0)
A = (1, 0)
B = (cos(a), sin(a))
geom.jana(P, A, "$r'$", 0.5 * 0.9, puoli = False)
geom.jana(P, B)
geom.kaari(P, 0.9, 0, a_deg, "$c'$", 0.5 * a_deg)
geom.kulma(A, P, B, r"$\alpha$")
\end{kuva}

Jos valitaan kaarenpituudeksi jokin toinen luku $r'$, saadaan toinen kaarenpituus $c'$, mutta edelleen kulman kylkien erottamat sektori on yhdenmuotoinen alkuperäisen kanssa, joten kaaren pituuden suhde säteeseen on molemmissa sektoreissa sama:
\[\frac{c}{r} = \frac{c'}{r'}.\]

Tästä seuraa, että millä tahansa säteen $r$ valinnalla suhde $\frac{c}{r}$ säilyy vakiona samalle kulmalle $\alpha$. Siis suhde $\frac{c}{r}$ on luonnollinen kulman suuruuden mitta, jota kutsumme radiaaniksi:

\begin{luoKuva}{radiaanimaaritelma}
skaalaa(3)
a = 0.24 * pi
a_deg = 180 * a / pi
P = geom.piste(0, 0)
A = (1, 0)
B = (cos(a), sin(a))
geom.jana(P, A, "$r$", 0.5, puoli = False)
geom.jana(P, B)
geom.kaari(P, 1, 0, a_deg, "$c$", 0.5 * a_deg)
geom.kulma(A, P, B, r"$\alpha$")
\end{luoKuva}

\laatikko{
{\bf Radiaanin määritelmä}

\raisebox{-.5\height}{\naytaKuva{radiaanimaaritelma}}
Kulman $\alpha$ suuruus radiaaneina on $\displaystyle\frac{r}{c}$.
}

Kulman suuruus radiaaneina on kahden käyränpituuden suhteena paljas luku, mutta monesti merkitään kulman suuruuden perään rad korostamaan sitä, että kyseessä on kulma.


