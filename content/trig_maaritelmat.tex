\section{Sini, kosini ja tangentti}
\begin{kuva}
A = (0, 0)
B = (4, 0)
C = (4, 2.5)
geom.jana(A, B, "$b$")
geom.jana(B, C, "$a$")
geom.jana(C, A, "$c$")
geom.kulma(B, A, C, r"$\alpha$")
geom.suorakulma(C, B, A)
\end{kuva}

Kaikki suorakulmiaiset kolmiot, joissa yksi terävä kulma on $\alpha$, ovat kk-säännön nojalla yhdenmuotoisia. Siis sivujen $a$, $b$ ja $c$ keskinäiset suhteet riippuvat vain kulmasta $\alpha$. Näille suhteille on annettu nimet sini, kosini ja tangentti
\[\sin(\alpha) = \frac{a}{c},\hspace{1cm}\cos(\alpha) = \frac{b}{c}, \hspace{1cm}\tan(\alpha) = \frac{a}{b}.\]

Tätä määritelmää voidaan kuitenkin yleistää koskemaan muitakin kuin teräviä kulmia. Valitaan kehäpiste yksikköympyrän kehältä siten, että origoon muodostuu kulma $\alpha$ positiivisen x-akselin ja valitun kehäpisteen välille. Tulkitaan kulma $\alpha$ suunnattuna, eli negatiivinen kulma kasvaa x-akselilta myötäpäivään ja positiivinen vastapäivään. Nyt määritellään sini ja kosini valitun kehäpisteen y- ja x-koordinaatteina. % FIXME: parempi selitys suunnatulle kulmalle.

% FIXME: piirrä suunnattu kulma nuolella
\begin{kuva}
skaalaa(4)
kuvaaja.pohja(-1, 1, -1, 1)
geom.ympyra((0, 0), 1)

O = (0, 0)
A = (1, 0)

a_deg = -55
a = pi * a_deg / 180

P = geom.piste(cos(a), sin(a), r"$(\cos(-55^\circ), \sin(-55^\circ))$")
geom.jana(O, P)
geom.kulma(P, O, A, "$-55^\circ$", kasvata = 0.15)

a = 0.8
a_deg = 180 * a / pi

P = geom.piste(cos(a), sin(a), r"$(\cos(0,8), \sin(0,8))$")
geom.jana(O, P)
geom.kulma(A, O, P, "$0,8\\text{ rad}$", kasvata = 0.3)

a = 2.6
a_deg = 180 * a / pi

P = geom.piste(cos(a), sin(a), r"$(\cos(\alpha), \sin(\alpha))$", suunta = 180)
geom.jana(O, P)
geom.kulma(A, O, P, r"$\alpha$", suunta = 70)
\end{kuva}
