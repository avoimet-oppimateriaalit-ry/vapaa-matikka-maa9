\section{Sini, kosini ja tangentti}
\begin{kuva}
A = (0, 0)
B = (4, 0)
C = (4, 2.5)
geom.jana(A, B, "$b$")
geom.jana(B, C, "$a$")
geom.jana(C, A, "$c$")
geom.kulma(B, A, C, r"$\alpha$")
geom.suorakulma(C, B, A)
\end{kuva}

Kaikki suorakulmiaiset kolmiot, joissa yksi terävä kulma on $\alpha$, ovat kk-säännön nojalla yhdenmuotoisia. Siis sivujen $a$, $b$ ja $c$ keskinäiset suhteet riippuvat vain kulmasta $\alpha$. Näille suhteille on annettu nimet sini, kosini ja tangentti
\[\sin(\alpha) = \frac{a}{c},\hspace{1cm}\cos(\alpha) = \frac{b}{c}, \hspace{1cm}\tan(\alpha) = \frac{a}{b}.\]

Jos valitaan $c = 1$, ja piirretään kulma $\alpha$ origoon siten, että suora kulma on positiivisella x-akselilla, huomataan että kolmas kärki on yksikköympyrän kehällä ja sen koordinaatit ovat $(b, a) = (\cos(\alpha), \sin(\alpha))$:

\begin{kuva}
skaalaa(3.5)
kuvaaja.pohja(-1, 1, -1, 1)
geom.ympyra((0, 0), 1)

O = (0, 0)
A = (1, 0)

a = 0.9
a_deg = 180 * a / pi

P = geom.piste(cos(a), sin(a), r"$(\cos(\alpha), \sin(\alpha))$")
with paksuus(2):
	geom.jana(P, O, "$c = 1$")
	geom.jana((P[0], 0), P, "$a$")
	geom.jana(O, (P[0], 0), "$b$")
geom.kulma(A, O, P, r"$\alpha$")
\end{kuva}

Jatketaan kolmion hypotenuusa suoraksi ja lasketaan sen ja suoran $x = 1$ leikkauspiste. Syntyvä kolmio $OP'Q'$ on yhdenmuotoinen kolmion $OPQ$ kanssa, eli
\[\frac{y}{1} = \frac{a}{b} = \tan(\alpha).\]
Siis tangentti voidaan laskea leikkauspisteen y-koordinaattina.

\begin{kuva}
skaalaa(3.5)
kuvaaja.pohja(-1, 1.2, -1, 1.5)
geom.ympyra((0, 0), 1)

O = geom.piste(0, 0, "$O$", suunta = -135)
Ap = geom.piste(1, 0, "$P'$", suunta = -45)

a = 0.9
a_deg = 180 * a / pi

B = geom.piste(cos(a), sin(a), r"$Q$")
A = geom.piste(B[0], 0, "$P$", suunta = -45)

Bp = geom.piste(1, B[1] / B[0], "$Q' = (1, y)$")

geom.jana(Bp, O)
geom.jana(A, B, "$a$")
geom.jana(O, A, "$b$")
geom.suora(Ap, Bp)
geom.jana(Ap, Bp, "$y$")
geom.kulma(A, O, B, r"$\alpha$")
\end{kuva}

Sini, kosini ja tangentti voitiin siis yhtäpitävästi määritellä yksikköympyrän kehäpisteen avulla. Näitä määritelmiä käyttäen voidaan määritellä sini, kosini ja tangentti myös muille kun teräville kulmille. Määritellään myös negatiivisen kulman sini kosini ja tangentti tulkitsemalla kulma suunnattuna kulmana. % FIXME: selitys suunnatulle kulmalle.

% FIXME: piirrä suunnattu kulma nuolella
% FIXME: yli 360 asteen kulmat
\begin{kuva}
skaalaa(3.5)
kuvaaja.pohja(-1, 1.2, -1.5, 1.2)
geom.ympyra((0, 0), 1)

O = (0, 0)
A = (1, 0)

a_deg = -55
a = pi * a_deg / 180

P = geom.piste(cos(a), sin(a), r"$(\cos(-55^\circ), \sin(-55^\circ))$")
Q = geom.piste(1, P[1] / P[0], r"$(1, \tan(-55^\circ))$")
geom.jana(O, Q)
geom.kulma(P, O, A, "$-55^\circ$", kasvata = 0.15)

a = 0.8
a_deg = 180 * a / pi

P = geom.piste(cos(a), sin(a), r"$(\cos(0,8), \sin(0,8))$")
Q = geom.piste(1, P[1] / P[0], r"$(1, \tan(0,8))$")
geom.jana(O, Q)
geom.kulma(A, O, P, "$0,8\\text{ rad}$", kasvata = 0.3)

a = 2.85
a_deg = 180 * a / pi

P = geom.piste(cos(a), sin(a), r"$(\cos(\alpha), \sin(\alpha))$", suunta = 180)
Q = geom.piste(1, P[1] / P[0], r"$(1, \tan(\alpha))$")
geom.jana(O, Q)
geom.jana(O, P)
geom.kulma(A, O, P, r"$\alpha$", suunta = 70)

geom.suora((1, 0), (1, 1))
\end{kuva}
