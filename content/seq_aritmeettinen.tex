\section{Aritmeettinen lukujono}

Lukujonoa, jonka peräkkäisten jäsenten erotus on aina sama, sanotaan \termi{artimeettinen lukujono}{aritmeettiseksi lukujonoksi}. Tätä peräkkäisten jäsenten erotusta sanotaan \termi{differenssi}{differenssiks}i $d$.

\laatikko{
	Aritmeettisen lukujonon $a_n$ jäsenille on voimassa ehto \(d=a_{n+1}-a_n\).
}

\laatikko{
	Aritmeettisen lukujonon yleinen jäsen $a_n$ saadaan kaavasta \( a_n = a_1+ (n-1)d \).
}

Esimerkiksi lukujono $3, 5, 7, 8, \ldots$ voi olla aritmettinen, joska sen peräkkäisten jäsenten erotus on aina 2. Erotus voi olla myös negatiivinen tai nolla kuten on tapaus jonoissa  $7, 6, 5, 4, \ldots$ ja  $3, 3, 3, 3, \ldots$.

\begin{esimerkki}

Kuinka mones jäsen luku 44 on aritmeettisessa lukujonossa $4, 12,  \ldots$?

Lukujonon ensimmäinen jäsen on $a_1=4$ ja differenssi $d=12-4=8$.

\begin{align*}
	a_n &= a_1+(n-1)d  \\
	44 &= 4 +(n-1)8  \\
	40 &= (n-1)8 & & | \, :8 \\
	5 &= n-1 \\
	6 &= n\\
\end{align*}

Vastaus: Kuudes jäsen.

\end{esimerkki}

\laatikko{
	Aritmeettisen lukujonon $n$ ensimmäisen jäsenen summa on \( S_n = n \frac{a_1+ a_n}{2} \).
}

