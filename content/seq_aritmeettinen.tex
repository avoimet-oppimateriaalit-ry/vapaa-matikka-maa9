\section{Aritmeettinen lukujono}

Lukujonoa, jonka peräkkäisten jäsenten erotus on aina sama, sanotaan \termi{artimeettinen lukujon}{aritmeettiseksi lukujonoksi}. Tätä peräkkäisten jäsenten erotusta sanotaan \termi{differenssi}{differenssiks}i $d$.

\laatikko{
	Aritmeettisen lukujonon $a_n$ jäsenille on voimassa ehto \(d=a_{n+1}-a_n\).
}

\laatikko{
	Aritmeettisen lukujonon yleinen jäsen $a_n$ saadaan kaavasta \( a_n = a_1+ (n-1)d \).
}
