\section{Ylioppilastehtäviä}

\begin{tehtava} (K2014/12)
	Funktion $f(x)$ derivaatan likiarvoja pisteessä $x_0$ voidaan laskea lausekkeen 
	\[\frac{f(x_0+h)-f(x_0)}{h}\] avulla, kun $h>0$ on pieni. Oletetaan, että $f(x)=\sin x$, 
	$x_0=0,5$ ja $h=10^{-p}$, kun $p=3,\ldots,10$. Mikä näistä $p$:n arvoista antaa parhaan 
	likiarvon luvulle $f'(x_0)$? Tehtävässä muuttujan $x$ yksikkö on radiaani.
		\begin{vastaus}
			Paras likiarvo saadaan, kun $p=7$. 
		\end{vastaus}
\end{tehtava}
%lisännyt jaakko viertiö 17.5.2014

\begin{tehtava} (K2013/2a)
	Laske funktion $f(x)=\sin(3x)$ derivaatan tarkka arvo kohdassa $x=\frac{\pi}{9}$.
		\begin{vastaus}
			$f'(\frac{\pi}{9})=\frac{3}{2}$
		\end{vastaus}
\end{tehtava}
%lisännyt jaakko viertiö 17.5.2014

\begin{tehtava} (K2013/2c)
	Kulma \alpha toteuttaa ehdot $-\frac{\pi}{2}<\alpha<\frac{\pi}{2}$ ja $\sin\alpha=\frac{1}{4}$.
	Määritä luvun $\cos\alpha$ tarkka arvo.
		\begin{vastaus}
			$\cos\alpha=\frac{\sqrt{15}}{4}$
		\end{vastaus}
\end{tehtava}
%lisännyt jaakko viertiö 17.5.2014

\begin{tehtava} (K2013/9)
	Ratkaise yhtälö $\cos(2x)+\cos(3x)=0$.
		\begin{vastaus}
			$x=(2n+1)\pi$ tai $x=\frac{2n-1}{5}\pi$, missä $n\in\mathbb{Z}$
		\end{vastaus}
\end{tehtava}
%lisännyt jaakko viertiö 17.5.2014

\begin{tehtava} (K2013/11)
	Millä muuttujan $x$ arvolla jono $\ln2, \; \ln(2^x-2), \; \ln(2^x+2)$ on aritmeettinen?
		\begin{vastaus}
			$x=\frac{\ln6}{\ln2}$
		\end{vastaus}
\end{tehtava}
%lisännyt jaakko viertiö 17.5.2014

\begin{tehtava} (S2012/4a)
	Olkoon $\alpha\in\left[\pi,\frac{3\pi}{2}\right]$ sellainen kulma, että $\cos\alpha=-\frac{1}{3}$.
	Määritä lukujen $\sin\alpha$ ja $\tan\alpha$ tarkat arvot.
		\begin{vastaus}
			$\sin\alpha=-\frac{2\sqrt{2}}{3}$, $\tan\alpha=2\sqrt{2}$
		\end{vastaus}
\end{tehtava}
%lisännyt jaakko viertiö 17.5.2014

\begin{tehtava} (S2012/9)
	Olkoot 
	\begin{align*}
	\bar{a}&=(\cos\varphi-2\sin\varphi)\bar{i}+\bar{j}+(\sin\varphi+2\cos\varphi)\bar{k}, \\
	\bar{b}&=(\cos\varphi+\sin\varphi)\bar{i}+\bar{j}+(\sin\varphi-\cos\varphi)\bar{k}.
	\end{align*}
	
	\alakohdat{
		§ Osoita, että vektorit $\bar{a}$ ja $\bar{b}$ ovat kohtisuorassa toisiaan vastaan 
		kaikilla $\varphi\in\mathbb{R}$.
		§ Olkoon $\varphi=0$. Onko olemassa sellaisia kertoimia $s,\,t\in\mathbb{R},$ että 
		$\bar{i}-\bar{j}=s\bar{a}+t\bar{b}$?
	}
		\begin{vastaus}
			\alakohdat{
				§ Pistetulolla.
				§ Tälläisiä kertoimia ei ole olemassa.
			}
		\end{vastaus}
%sopiiko tämä tänne? jaakkoviertiö 17.5.2014
\end{tehtava}


\begin{tehtava} (S2012/11)
	\alakohdat{
		§ Geometrisen jonon kaksi peräkkäistä termiä ovat rationaalilukuja. Osoita, että 
		jonon kaikki termit ovat rationaalilukuja.
		§ Geometrisessa jonossa on ainakin kaksi rationaalista termiä. Osoita, että 
		rationaalisia termejä on äärettömän monta.
	}
\end{tehtava}
%lisännyt jaakko viertiö 17.5.2014

\begin{tehtava} (K2012/2d)
	Sievennä lauseke $\sin^2x+\cos^2(x+2\pi)$.
		\begin{vastaus}
			$\sin^2x+\cos^2(x+2\pi)=1$
		\end{vastaus}
\end{tehtava}
%lisännyt jaakko viertiö 17.5.2014

\begin{tehtava} (K2012/10)
	Ratkaise yhtälöt
	\alakohdat{
		§ $3\tan\frac{x}{2}+3=0$
		§ $2\sin^2x+3\cos x-3=0$
	}
		\begin{vastaus}
			\alakohdat{
				§ $x=-\frac{\pi}{2}+2n\pi, n\in\mathbb{Z}$.
				§ $x=\pm\frac{\pi}{3}+2n\pi$ tai $x=2n\pi, n\in\mathbb{Z}$
			}
		\end{vastaus}
\end{tehtava}
%lisännyt jaakko viertiö 17.5.2014

\begin{tehtava} (S2011/10)
	Määritä funktion \[f(x)=3\cos^2x-\sin^2x-2\] nollakohdat sekä suurin ja pienin arvo.
		\begin{vastaus}
			Nollakohdat ovat $\pm\frac{\pi}{6}+2n\pi$ ja $\pm\frac{5\pi}{6}+2n\pi, 
			n\in\mathbb{Z}$. Suurin arvo on 1 ja pienin -3.
		\end{vastaus}
\end{tehtava}
%lisännyt jaakko viertiö 17.5.2014

\begin{tehtava} (S2010/7)
	Määritä funktion \[f(x)=\cos x-\frac{1}{2}\cos 2x\] suurin ja pienin arvo. Missä pisteissä suurin 
	arvo saavutetaan?
		\begin{vastaus}
			Funktion suurin arvo on $\frac{3}{4}$ ja pienin arvo on $-\frac{3}{2}$.
			Maksimikohdat ovat $x=\pm\frac{\pi}{3}+2n\pi, n\in\mathbb{Z}$. 
		\end{vastaus}
\end{tehtava}
%lisännyt jaakko viertiö 17.5.2014

\begin{tehtava} (S2010/8)
	Jono $(a_n)$ on aritmeettinen jono. Osoita, että jono $(b_n)$, missä $b_n=3^{a_n}$, on geometrinen. 
	Millä jonoa $(a_n)$ koskevalla ehdolla jono $(b_n)$ on aidosti vähenevä?
		\begin{vastaus}
			Jos $a_n=a+nd$, niin $b_n=3^a\cdot(3^d)^n$. Jono $(b_n)$ on aidosti vähenevä, jos 
			$3^d<1$ eli $d<0$.
		\end{vastaus}
\end{tehtava}
%lisännyt jaakko viertiö 17.5.2014

\begin{tehtava} (K2010/2b)
	Derivoi funktio $f(x)=x\sin x$.
		\begin{vastaus}
			$f'(x)=\sin x+x\cos x$
		\end{vastaus}
\end{tehtava}
%lisännyt jaakko viertiö 17.5.2014

\begin{tehtava} (K2010/9)
	Tutki, kuinka monta juurta yhtälöllä \[3\tan x-1=4x\] on välillä 
	$\left]-\frac{\pi}{2},\frac{\pi}{2}\right[$.
		\begin{vastaus}
			Yksi.
		\end{vastaus}
\end{tehtava}
%lisännyt jaakko viertiö 17.5.2014

\begin{tehtava} (K2008/3b)
	Laske funktion $f(x)=\frac{2+\sin x}{2+\cos x}$ derivaatta pisteessä $x=\frac{\pi}{2}$.
		\begin{vastaus}
			$f'(\frac{\pi}{2})=\frac{3}{4}$
		\end{vastaus}
\end{tehtava}
%lisännyt jaakko viertiö 17.5.2014

\begin{tehtava} (S2007/2a)
	Olkoon $f(x)=\sin x\cos x$. Laske $f'(0)$.
		\begin{vastaus}
			$f'(0)=1$
		\end{vastaus}
\end{tehtava}
%lisännyt jaakko viertiö 17.5.2014

\begin{tehtava} (S2007/14*)
	Osoita, että funktio $f \colon \mathbb{R}\to\mathbb{R}, f(x)=x-\cos x$, on aidosti kasvava ja 
	että se saa kaikki reaalilukuarvot. Päättele, että tällöin yhtälöllä $f(x)=0$ on vain yksi 
	ratkaisu, ja määritä se kolmen desimaalin tarkkuudella.
		\begin{vastaus}
			Funktion nollakohta kolmen desimaalin tarkkuudella on 0,739.
		\end{vastaus}
\end{tehtava}
%lisännyt jaakko viertiö 17.5.2014

\begin{tehtava} (K2007/3b)
	Mikä on vuotuinen korkoprosentti, jos tilille talletettu rahamäärä kasvaa korkoa korolle 
	1,5-kertaiseksi 10 vuodessa? Lähdeveroa ei oteta huomioon. Anna vastaus prosentin sadasosan 
	tarkkuudella.
		\begin{vastaus}
			$4,14\%$
		\end{vastaus}
\end{tehtava}
%lisännyt jaakko viertiö 17.5.2014

\begin{tehtava} (S2006/6)
	Määritä funktion $f(x)=\cos^2x+\sin x$ suurin ja pienin arvo.
		\begin{vastaus}
			Funktion suurin arvo on $\frac{5}{4}$ ja pienin arvo $-1$.
		\end{vastaus}
\end{tehtava}
%lisännyt jaakko viertiö 17.5.2014

\begin{tehtava} (K2006/5)
	Millä vakion $a$ arvoilla yhtälöllä $\sin x=5-a^2\sin x$ on ratkaisuja?
		\begin{vastaus}
			Arvoilla $a\leq-2$ ja $a\geq2$
		\end{vastaus}
\end{tehtava}
%lisännyt jaakko viertiö 17.5.2014

\begin{tehtava} (K2005/2b)
	Tiedetään, että $\sin x=-\frac{1}{\sqrt{5}}$ ja $180^{\circ}<x<270^{\circ}$. Määritä $\cos x$ 
	ja $\tan x$ (tarkat arvot).
		\begin{vastaus}
			$\cos x=-\frac{2}{\sqrt{5}}$ \; ja \; $\tan x=\frac{1}{2}$
		\end{vastaus}
\end{tehtava}
%lisännyt jaakko viertiö 18.5.2014

\begin{tehtava} (S2004/8)
	Olkoon annettuna trigonometrian kaavat 
		\alakohdat{
			§ $\sin^2\alpha+\cos^2\alpha=1$
			§ $\sin2\alpha=2\sin\alpha\cos\alpha$
			§ $\cos2\alpha=\cos^2\alpha-\sin^2\alpha$
			§ $\tan\alpha=\frac{\sin\alpha}{\cos\alpha}$.
		}
	Osoita pelkästään näiden perusteella oikeiksi seuraavat kaavat: 
	\[\sin x=\frac{2\tan\frac{x}{2}}{1+\tan^2\frac{x}{2}}, \qquad \cos x=\frac{1-\tan^2\frac{x}{2}}{1+\tan^2\frac{x}{2}}.\]
	Ilmoita, mitä kaavaa olet missäkin laskun vaiheessa käyttänyt.
\end{tehtava}
%lisännyt jaakko viertiö 18.5.2014

\begin{tehtava} (S2003/6)
	Määritä $\sin(x-y)$, kun $\sin x=\frac{1}{4}, -\frac{\pi}{2}\leq x \leq \frac{\pi}{2}$ ja $\cos y=-\frac{1}{3}, 
	\pi \leq y \leq 2\pi$. Tarkka arvo ja kaksidesimaalinen likiarvo.
		\begin{vastaus}
			$\sin(x-y)=\frac{2\sqrt{30}-1}{12}\approx0,83$
		\end{vastaus}
\end{tehtava}
%lisännyt jaakko viertiö 18.5.2014

\begin{tehtava} (S2003/12)
	Isä tallettaa poikansa tilille joka kuukauden alussa 200 \euro \, vuodenvaihteessa tapahtuneesta syntymästä alkaen.
	Tilille maksetaan 1,5\,\% vuotuista korkoa, joka liitetään pääomaan aina vuoden lopussa.
		\alakohdat{
			§ Kuinka paljon rahaa tilillä on, kun poika täyttää 18 vuotta?
			§ Kuinka kauan isän olisi talletettava, jotta tilillä olisi rahaa kaksiota varten, 
			kun kaksion hinnaksi oletetaan 135\,000 \euro ?
		}
			\begin{vastaus}
				\alakohdat{
					  § 49\,574,04 \euro
					  § 41 vuotta
				}
			\end{vastaus}
\end{tehtava}
%lisännyt jaakko viertiö 18.5.2014

\begin{tehtava} (K2001/6)
	Määritä funktion \[f(x)=\frac{5}{4+3\cos2x}\] suurin ja pienin arvo reaalilukujen joukossa. Millä muuttujan $x$ 
	arvoilla nämä saadaan?
		\begin{vastaus}
			Funktion suurin arvo on 5, ja se saavutetaan, kun $x=\frac{\pi}{2}+n\pi, n\in\mathbb{Z}$.
			Funktion pienin arvo on $\frac{5}{7}$ ja se saavutetaan, kun $x=n\pi, n\in\mathbb{Z}$.
		\end{vastaus}
\end{tehtava}
%lisännyt jaakko viertiö 18.5.2014

\begin{tehtava} (K2001/11)
	Pallon sisään asetetaan kuutio, jonka kärjet ovat pallon pinnalla. Kuution sisään asetetaan pallo, joka sivuaa 
	jokaista kuution sivutahkoa. Tämän sisään asetetaan jälleen kuutio jne. Osoita, että pallojen 
		\alakohdat{
			§ säteet
			§ pinta-alat
			§ tilavuudet
		}
	kukin erikseen muodostavat geometrisen jonon. Määritä jonojen suhdeluvut.
		\begin{vastaus}
			\alakohdat{
				§ Suhdeluku on $\frac{1}{\sqrt{3}}$
				§ Suhdeluku on $\frac{1}{3}$
				§ Suhdeluku on $\frac{1}{3\sqrt{3}}$
			}
		\end{vastaus}
\end{tehtava}
%lisännyt jaakko viertiö 18.5.2014

\begin{tehtava} (S1999/4a)
	Määritä ne reaaliluvut $x$, joilla funktion $f(x)=\cos(\frac{\pi}{2}-x)$ derivaatta on $f'(x)=\sin x$.
		\begin{vastaus}
			$x=\pi+4n\pi$
		\end{vastaus}
\end{tehtava}
%lisännyt jaakko viertiö 18.5.2014
%vastaus kannattaa tarkistaa... -jv

\begin{tehtava} (S1998/8a)
	Osoita, että yhtälöllö $5\tan x-3=10x$ on välillä $\left]-\frac{\pi}{2}\,,\,\frac{\pi}{2}\right[$ täsmälleen yksi juuri. 
	Määritä tämän jälkeen juuren arvo yhden desimaalin tarkkuudella.
\end{tehtava}
%lisännyt jaakko viertiö 18.5.2014
%numeeriset menetelmät vasta kurssissa 12 mutta minusta haarukointi on jo derivaattakurssissa eli 7-kurssissa. jv

\begin{tehtava} (S1997/6b)
	Todista oikeaksi kaava $f(x)=\cos^6x+\sin^6x=1-\frac{3}{4}\sin^22x$ ja määritä tämän kaavan avulla funktion $f$ 
	suurin ja pienin arvo.
\end{tehtava}
%lisännyt jaakko viertiö 18.5.2014

\begin{tehtava} (S1996/6)
	Määritä funktion $f(x)=2\sin x+\cos2x$ suurin ja pienin arvo.
\end{tehtava}
%lisännyt jaakko viertiö 18.5.2014

\begin{tehtava} (S1996/8a)
	Määritä lukujonon $(a_n)$, missä $a_n=\frac{n^4}{2^n} \; (n=1,2,3,\ldots)$ suurin luku. Perustele vastauksesi.
\end{tehtava}
%lisännyt jaakko viertiö 18.5.2014

\begin{tehtava} (S1994/3)
	Laske funktion $f(x)=\sin2x$ tarkka arvo, kun $\sin x=-\frac{8}{17}$ ja $\pi<x<\frac{3\pi}{2}$.
\end{tehtava}
%lisännyt jaakko viertiö 18.5.2014

\begin{tehtava} (K1994/3a)
	Ratkaise yhtälö $2\cos^2x-3\cos x-2=0$
\end{tehtava}
%lisännyt jaakko viertiö 18.5.2014

\begin{tehtava} (K1993/6a)
	Tutki, onko yhtälöllä $\sin x+\cos x=1,41422$ reaalijuuria.
\end{tehtava}
%lisännyt jaakko viertiö 18.5.2014

\begin{tehtava} (S1992/1)
	Määritä vakiot $a$ ja $b$ siten, että funktio $f(x)=a\sin x+b\cos x$ toteuttaa ehdot $f(\frac{\pi}{2})=\pi$ 
	ja $f'(\frac{\pi}{2})=2$.
\end{tehtava}
%lisännyt jaakko viertiö 18.5.2014

\begin{tehtava} (S1992/3b)
	Ratkaise yhtälö $\sin x=\cos(2x+\frac{\pi}{2})$.
\end{tehtava}
%lisännyt jaakko viertiö 18.5.2014

\begin{tehtava} (K1992/9b)
	Lausu funktio $f(x)=12\cos x-5\sin x$ muodossa $R\cos(x+\omega)$, missä $R$ ja $\omega$ ovat vakioita, ja 
	määritä tätä tietä funktion $f$ suurin ja pienin arvo. Millä $x$:n arvoilla ne saavutetaan?
\end{tehtava}
%lisännyt jaakko viertiö 18.5.2014

\begin{tehtava} (K1973/1)
	Määritä vakiot $a$ ja $b$ siten, että funktio $f(x)=a\sin x+b$ täyttää ehdot $f(0)=1$ ja $f'(0)=2$.
\end{tehtava}
%lisännyt jaakko viertiö 18.5.2014

\begin{tehtava} (K1973/8)
	Määritä funktion $f(x)=e^{-x}\sin x$, missä $x\geq0$, ääriarvot. Osoita, että funktion peräkkäiset maksimiarvot muodostavat 
	geometrisen jonon (geometrisen sarjan).
\end{tehtava}
%lisännyt jaakko viertiö 18.5.2014

\begin{tehtava} (S1972/4)
	Ratkaise yhtälö \[\frac{\sin2x}{\sin^2x+\cos2x}=2\]
\end{tehtava}
%lisännyt jaakko viertiö 18.5.2014

\begin{tehtava} (S1971/5)
	Osoita, että funktio $f(x)=\cos\alpha-\cos(2x+\alpha)$, missä $\alpha$ on vakio, saa kuvaajansa kaikissa käännepisteissä 
	saman arvon (käännepisteellä tarkoitetaan derivaatan nollakohtaa).
\end{tehtava}
%lisännyt jaakko viertiö 18.5.2014

\begin{tehtava} (S1970/3)
	Tee jompikumpi seuraavista tehtävistä:
		\alakohdat{
			§ Sievennä lauseke $\sin^2\alpha+\sin^2(\alpha+\frac{2\pi}{3})+\sin^2(\alpha-\frac{2\pi}{3})$
			yksinkertaisimpaan muotoonsa.
			§ Laske vektoreiden $\bar{u}=\cos\alpha\,\bar{i}+\sin\alpha\,\bar{j}, \; 
			\bar{v}=\cos(\alpha+\frac{2\pi}{3})\,\bar{i}+\sin(\alpha+\frac{2\pi}{3})\,\bar{j}$ \; ja \; 
			$\bar{w}=\cos(\alpha-\frac{2\pi}{3})\,\bar{i}+\sin(\alpha-\frac{2\pi}{3})\,\bar{j}$ summa.
		}
\end{tehtava}
%lisännyt jaakko viertiö 18.5.2014

\begin{tehtava} (K1970/9)
	Osoita, että $\tan x<2x$ kun $0<x<\frac{\pi}{3}$.
\end{tehtava}
%lisännyt jaakko viertiö 18.5.2014
