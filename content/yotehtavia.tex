\section{Ylioppilastehtäviä}

\begin{tehtava} (K2014/12)
	Funktion $f(x)$ derivaatan likiarvoja pisteessä $x_0$ voidaan laskea lausekkeen 
	\[\frac{f(x_0+h)-f(x_0)}{h}\] avulla, kun $h>0$ on pieni. Oletetaan, että $f(x)=\sin x$, 
	$x_0=0,5$ ja $h=10^{-p}$, kun $p=3,\ldots,10$. Mikä näistä $p$:n arvoista antaa parhaan 
	likiarvon luvulle $f'(x_0)$? Tehtävässä muuttujan $x$ yksikkö on radiaani.
\end{tehtava}

\begin{tehtava} (K2013/2a)
	Laske funktion $f(x)=\sin(3x)$ derivaatan tarkka arvo kohdassa $x=\frac{\pi}{9}$.
\end{tehtava}

\begin{tehtava} (K2013/2c)
	Kulma \alpha toteuttaa ehdot $-\frac{\pi}{2}<\alpha<\frac{\pi}{2}$ ja $\sin\alpha=\frac{1}{4}$.
	Määritä luvun $\cos\alpha$ tarkka arvo.
\end{tehtava}

\begin{tehtava} (K2013/9)
	Ratkaise yhtälö $\cos(2x)+\cos(3x)=0$.
\end{tehtava}

\begin{tehtava} (K2013/11)
	Millä muuttujan $x$ arvolla jono $\ln2, \; \ln(2^x-2), \; \ln(2^x+2)$ on aritmeettinen?
\end{tehtava}

\begin{tehtava} (S2012/4a)
	Olkoon $\alpha\in\left[\pi,\frac{3\pi}{2}\right]$ sellainen kulma, että $\cos\alpha=-\frac{1}{3}$.
	Määritä lukujen $\sin\alpha$ ja $\tan\alpha$ tarkat arvot.
\end{tehtava}

\begin{tehtava} (S2012/9)
	Olkoot 
	\begin{align*}
	\bar{a}&=(\cos\varphi-2\sin\varphi)\bar{i}+\bar{j}+(\sin\varphi+2\cos\varphi)\bar{k}, \\
	\bar{b}&=(\cos\varphi+\sin\varphi)\bar{i}+\bar{j}+(\sin\varphi-\cos\varphi)\bar{k}.
	\end{align*}
	
	\begin{alakohdat}
		\alakohta{Osoita, että vektorit $\bar{a}$ ja $\bar{b}$ ovat kohtisuorassa toisiaan vastaan 
		kaikilla $\varphi\in\mathbb{R}$.}
		\alakohta{Olkoon $\varphi=0$. Onko olemassa sellaisia kertoimia $s,\,t\in\mathbb{R},$ että 
		$\bar{i}-\bar{j}=s\bar{a}+t\bar{b}$?}
	\end{alakohdat}
%sopiiko tämä tänne? jaakkoviertiö 17.5.2014
\end{tehtava}


\begin{tehtava} (S2012/11)
	\begin{alakohdat}
		\alakohta{Geometrisen jonon kaksi peräkkäistä termiä ovat rationaalilukuja. Osoita, että 
		jonon kaikki termit ovat rationaalilukuja.}
		\alakohta{Geometrisessa jonossa on ainakin kaksi rationaalista termiä. Osoita, että 
		rationaalisia termejä on äärettömän monta.}
	\end{alakohdat}
\end{tehtava}

\begin{tehtava} (K2012/2d)
	Sievennä lauseke $\sin^2x+\cos^2(x+2\pi)$.
\end{tehtava}

\begin{tehtava} (K2012/10)
	Ratkaise yhtälöt
	\begin{alakohdat}
		\alakohta{$3\tan\frac{x}{2}+3=0$}
		\alakohta{$2\sin^2x+3\cos x-3=0$}
	\end{alakohdat}
\end{tehtava}

\begin{tehtava} (S2011/10)
	Määritä funktion \[f(x)=3\cos^2x-\sin^2x-2\] nollakohdat sekä suurin ja pienin arvo.
\end{tehtava}

\begin{tehtava} (S2010/7)
	Määritä funktion \[f(x)=\cos x-\frac{1}{2}\cos 2x\] suurin ja pienin arvo. Missä pisteissä suurin 
	arvo saavutetaan?
\end{tehtava}

\begin{tehtava} (S2010/8)
	Jono $(a_n)$ on aritmeettinen jono. Osoita, että jono $(b_n)$, missä $b_n=3^{a_n}$, on geometrinen. 
	Millä jonoa $(a_n)$ koskevalla ehdolla jono $(b_n)$ on aidosti vähenevä?
\end{tehtava}

\begin{tehtava} (K2010/2b)
	Derivoi funktio $f(x)=x\sin x$.
\end{tehtava}

\begin{tehtava} (K2010/9)
	Tutki, kuinka monta juurta yhtälöllä \[3\tan x-1=4x\] on välillä 
	$\left]-\frac{\pi}{2},\frac{\pi}{2}\right[$.
\end{tehtava}

\begin{tehtava} (K2008/3b)
	Laske funktion $f(x)=\frac{2+\sin x}{2+\cos x}$ derivaatta pisteessä $x=\frac{\pi}{2}$.
\end{tehtava}

\begin{tehtava} (S2007/2a)
	Olkoon $f(x)=\sin x\cos x$. Laske $f'(0)$.
\end{tehtava}

\begin{tehtava} (S2007/14*)
	Osoita, että funktio $f \colon \mathbb{R}\to\mathbb{R}, f(x)=x-\cos x$, on aidosti kasvava ja 
	että se saa kaikki reaalilukuarvot. Päättele, että tällöin yhtälöllä $f(x)=0$ on vain yksi 
	ratkaisu, ja määritä se kolmen desimaalin tarkkuudella.
\end{tehtava}

\begin{tehtava} (K2007/3b)
	Mikä on vuotuinen korkoprosentti, jos tilille talletettu rahamäärä kasvaa korkoa korolle 
	1,5-kertaiseksi 10 vuodessa? Lähdeveroa ei oteta huomioon. Anna vastaus prosentin sadasosan 
	tarkkuudella.
\end{tehtava}

\begin{tehtava} (S2006/6)
	Määritä funktion $f(x)=\cos^2x+\sin x$ suurin ja pienin arvo.
\end{tehtava}

\begin{tehtava} (K2006/5)
	Millä vakion $a$ arvoilla yhtälöllä $\sin x=5-a^2\sin x$ on ratkaisuja?
\end{tehtava}
