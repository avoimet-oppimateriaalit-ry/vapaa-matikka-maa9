\section{Geometrinen lukujono}

Lukujonoa, jonka peräkkäisten jäsenten osamäärä on aina sama, sanotaan \termi{geometrinen lukujon}{geometriseksi lukujonoksi}. Tätä peräkkäisten jäsenten osamäärää merkitään yleensä kirjaimella $q$.

\laatikko{
Geometrinen lukujono $a_n$ jäsenille on voimassa ehto \( q = \frac{a_{n+1}}{a_n} \).
}

\laatikko{
	Geometrisen lukujonon yleinen jäsen $a_n$ saadaan kaavasta \( a_n = a_1q^{n-1} \).
}

Esimerkiksi lukujono $3, 6, 12, 24, \ldots$ voi olla aritmettinen, joska sen peräkkäisten jäsenten osamäärä on aina 2. Osamäärä voi olla myös negatiivinen jonoissa  $7, -7, 7, -7, \ldots$ ja  $ -4, 2, -1, \frac{1}{2}, \ldots$.

\begin{esimerkki}

Kuinka mones jäsen luku 972 on geometrisessa lukujonossa $4, 12,  \ldots$?

Lukujonon ensimmäinen jäsen on $a_1=4$ ja suhdeluku $q=\frac{12}{4}=3$.

\begin{align*}
	a_n &= a_1q^{n-1}  \\
	972 &= 4 \cdot 3^{n-1}  \\
	4 \cdot 3^{n-1} &= 972 & & | \, :4 \\
	3^{n-1} &= 243 & & | \, \ln\\
	\ln 3^{n-1} &= \ln 243 \\
	(n-1)\ln 3 &= \ln 243  & & | \, : \ln 3 \\
	n-1 &= frac{\ln 243}{\ln 3} \\
	n &= frac{\ln 243}{\ln 3} +1\\
	n &= 6\\
\end{align*}

Vastaus: Kuudes jäsen.

\end{esimerkki}