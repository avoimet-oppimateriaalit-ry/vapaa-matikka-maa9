\section{Geometrinen lukujono}

Lukujonoa, jonka peräkkäisten jäsenten osamäärä on aina sama, sanotaan \termi{geometrinen lukujon}{geometriseksi lukujonoksi}. Tätä peräkkäisten jäsenten osamäärää merkitään yleensä kirjaimella $q$.

\laatikko{
Geometrinen lukujono $a_n$ jäsenille on voimassa ehto \( q = \frac{a_{n+1}}{a_n} \).
}

\laatikko{
	Geometrisen lukujonon yleinen jäsen $a_n$ saadaan kaavasta \( a_n = a_1q^{n-1} \).
}